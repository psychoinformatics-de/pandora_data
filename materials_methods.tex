\section*{Materials and methods} 
Please provide a detailed account of the methods and materials used for the study in order to allow replication by other researchers.
Please include \begin{itemize}
\item 
the source of all samples, reagents, antibodies etc.
\item 
 how samples were selectedwhat exclusions were made, if any;
\item 
what was being measured
\item 
for processed data, any software used to process the data and, where possible, the source code should be made openly available.\item 

allowances made, if any, for controlling bias or unwanted sources of variability.
Limitations of the datasets.\item 

acronyms and abbreviations must be explained
\end{itemize}

Detailed description of the observation setup. QA statistics do not go here, but the "Dataset validation" section.

\subsection*{Participants}

Acquisition of the data described herein was part of a previously published
study \cite{Hanke_2014}, and took place in close temporal proximity (no more
than a few weeks apart). The participants in this data release are identical to
those previously reported \cite{Hanke_2014}.  They were fully instructed about
the nature of the study and were paid a total of 100 EUR for their
participation, which included the previously reported data acquisitions, as
well as the one described herein. All data acquisitions were jointly approved
by the ethics committee of the Otto-von-Guericke-University of Magdeburg,
Germany.


\subsection*{Stimulus}

All stimuli employed in this study are identical to those used in a previous
study \cite[for details refer to][]{Casey_2012}. They were five natural,
stereo, high-quality music stimuli for each of five different musical genres:
1) Ambient, 2) Roots Country 3) Heavy Metal, 4) 50s Rock'n'Roll, and 5)
Symphonic.  Each stimulus was a six second excerpt from the center of each
musical pieces.  Excerpts were normalized so that their RMS values were equal,
and a 50ms quarter-sine ramp was applied at the start and end of each excerpt
to suppress transients.

\subsection*{Procedures and stimulation setup}

The setup for audio-visual presentation was as previously reported
\cite{Hanke_2014}. Participants listened to the audio using custom-built in-ear
headphones that were driven by an MR confon mkII+ fed from an Aureon 7.1 USB
(Terratec) sound card through an optical connection.
Visual instructions were presented with an LCD projector (DLA-G150CL, JVC Ltd.)
on a rear-projection screen positioned behind the head coil within the magnetic
bore. Participants viewed the screen through a mirror attached to the head
coil.

At the start of each recording session, during the preparatory MR scans,
participants listened to a series of longer excerpts of musical pieces and
songs from the five different genres. During this phase participants were
instructed to request adjustments of the stimulus volume in order to guarantee
optimal perception of the stimuli against the noise pedestal emitted by the
scanner. The was no overlap between the songs presented in this phase and those
used as stimuli in the main experiment.

Eight scanning runs followed the initial sound calibration. Each run was
started by the participant with a key-press ready signal. Each run comprised 25
trials, five stimuli for each of the five runs. These 25 stimuli were identical
across runs and presented exactly once per run. The onset of each stimulus was
synchronized with the volume acquisition trigger signal emitted by the scanner.
Order of stimulus genres within each run was counter-balanced using De Bruijn
cycles \cite{Aguirre_2011} (alphabet size = 5, counter-balancing level = 2),
hence each genre was followed by any other genre equally often and exactly
once.  For each of the eight runs a unique De Bruijn cycle was generated.
However, the same set of 8 cycles was used for all participants in the study,
while randomizing the order of run sequences across participants. This was done
in order to maintain synchronous stimulation across participants for the
application of the hyperalignment algorithm. At the end of each run
participants were given the opportunity for break of variable length until the
indicated readiness for the next run. Most participants started the next run
within a minute.

Each run started with a white fixation cross in the center of the screen. With
the beginning of each trial, synchronous with the start of the stimulus, the
fixation cross turned green to indicated music playback, and changed back to
white after the playback had stopped.  In order to further disentangle
responses to individual genres, each trial was followed by a variable delay of
4, 6, or 8 seconds. For each of the five trials for any given genre, the 4 and
8 second delay occurred once, while the remained three trial had a 6 second
delay period. The order of delay lengths was randomized within a run's
stimulation sequence. During trials with an 8 second delay subjects were
presented with a yes/no question that replaced the fixation cross four seconds
after the end of the music stimulus. The content of the question was randomized
and asked for particular features of the stimulus that had just ended (``Was
there a female singer?'', ``Did the song have a happy melody?'', etc.).
Participants had to indicated their response by pressing one of two buttons
with the index or middle finger of their right hand corresponding to the
response alternative presented on the screen. ``Yes'' was always mapped to the
left side (index finger), ``No'' always to the right side (middle finger).  The
question had the purpose to keep the participants attentive to the stimuli and
counteract the affect of increasing familiarity across multiple runs.  This
stimulation procedure guaranteed a minimum of four second of uniform
stimulation (no audio stimulus, no visual stimulus other than a white fixation
cross) after each musical stimulus.

Stimulus presentation and response logging were implemented using PsychoPy
(CITE).  The source code of the complete implementation of the experiment, as
well as the actual stimulus sequences used for each run are available as
Supplementary Material in the data release. PsychoPy was running on a computer
with the (Neuro)Debian operating system (CITE).

\subsection*{Functional MRI data acquisition}

The acquisition protocol for functional MRI was largely identical with the one
previously reported \cite{Hanke_2014}, hence only differences and key facts are
listed here.

Importantly, the same landmark-based procedure for automatic slice positioning
that was used to align the scanner field-of-view between acquisition sessions,
was used again to align the field-of-view for this acquisition with the one in
the previous study \cite{Hanke_2014}. As the exact same alignment target was
used, this led to a very similar field-of-view configuration across
acquisitions.

Each acquisition run consisted of 153 volumes (repetition time of 2.0 seconds
with no inter-volume gaps), for a total of eight runs.

\subsection*{Physiological recordings}

The cardiac and respiratory trace was recorded for the full duration of all
eight runs. The acquisition setup for physiological was identical with the one
previously reported \cite{Hanke_2014}.
