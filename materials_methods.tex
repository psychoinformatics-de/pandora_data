c\section*{Materials and methods} 
Please provide a detailed account of the methods and materials used for the study in order to allow replication by other researchers.
Please include \begin{itemize}
\item 
the source of all samples, reagents, antibodies etc.
\item 
 how samples were selectedwhat exclusions were made, if any;
\item 
what was being measured
\item 
for processed data, any software used to process the data and, where possible, the source code should be made openly available.\item 

allowances made, if any, for controlling bias or unwanted sources of variability.
Limitations of the datasets.\item 

acronyms and abbreviations must be explained
\end{itemize}

Detailed description of the observation setup. QA statistics do not go here, but the "Dataset validation" section.

\subsection*{Participants}
Acquisition of the data described herein was part of the study first published in \cite{Hanke_2014}, and took place in close temporal proximity (no more than a few weeks apart). The participants in this data release are the identical to those reported in \cite{Hanke_2014}.
They were, again, fully instructed about the nature of the study, and gave their informed consent for participation in the study as well as for publicly sharing all obtained data in anonymized form. They were paid a total of 100 EUR for their participation, which included the previously reported data acquisitions, as well as the one described herein. All data acquisitions were jointly approved by the ethics committee of the Otto-von-Guericke-University of Magdeburg, Germany.


\subsection*{Stimulus}

\subsection*{Procedures and stimulation setup}

\subsection*{Functional MRI data acquisition}
The acquisition protocol for functional MRI was largely identical with the one reported in \cite{Hanke_2014}, hence only differences and key facts are listed here.

Importantly, the same landmark-based procedure for automatic slice positioning that was used in \cite{Hanke_2014} to align the scanner field-of-view between acquisition sessions, was used again to align the field-of-view for this acquisition with the one of \cite{Hanke_2014}. As the exact same alignment target was used, this led to a very similar field-of-view configuration across acquisitions.

Each acquisition run consisted of 153 volumes, for a total of eight runs.

\subsection*{Physiological recordings}
The cardiac and respiratory trace was recorded for the full duration of all eight runs. The acquisition setup for physiological was identical with the one reported in \cite{Hanke_2014}.
